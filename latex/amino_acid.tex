\chapter{Highly Patterned Ontologies}
\label{cha:highly-patt-ontol}

Many ontologies contain patterns---that is collections of classes and
properties which occur repetitive through the ontology. Dealing with this in
many ontology development environments is painful. Tawny-OWL is a fully
programmatic environment, however. Patterns are dealt with by writing
functions and passing parameters; in otherwords, the same way that we deal
with code duplication more generally.

In this Chapter, we will first explore how to use patterns than
Tawny-OWL provides explicit support for; then we will move on to show
how to modify and extend these patterns in an ontology specific
way. Finally, we will show how to use the fully programmatic
capabilies of Tawny-OWL to generate a large number of classes in a way
that is unique to one ontology.

\section{Dealing with Patterns}
\label{sec:deal-with-patt}

Some ontologies have very few patterns; all the classes and objects are
unique. These ontologies tend to be very small, however. Most ontologies
describe many similar things with just a few details differing between them.
In this chapter, we use the amino-acid ontology -- this describes the chemical
constituants that make up proteins. There are twenty of these and they are all
very similar, with the same properties.

Graphical tools can provide a partial solution to this problem, by supporting
the building of these patterns. For instance, Protege had ``wizards'' to build
various patterns. In fact, the first version of the amino-acid ontology was
built to demonstrate one of these patterns \cite{todo}. This requires
extension of the editor for every new pattern, which is acceptable for some
generic patterns which can be widely reused, but works badly for patterns with
a narrow scope.

An alternative is to use a language like
OPPL~\cite{aranguren_Stevens_Antezana_2009}, which can directly specify
patterns and transformations to ontologies. However, this requires the use of
two syntaxes or environments -- one for ``normal'' ontological code, and one
for patternised. It also presents a maintainence problem -- the normal and
patternised code is intertwinned, so updating a pattern is difficult.

Tawny-OWL take an alternative approach. Instead of providing an
alternative language like OPPL, all ontological statements are written
in Clojure, which is, itself, an full programming languages. Patterns
can be built straight-forwardly by writing or using functions; this
can be done in a general library for generic patterns, shared between
ontologies. Or, alternatively, it can be done specifically for
individual ontologies, in the same syntax, files and development
environment as the normal parts. Updates cease to be a problem; in the
worst case scenario, this requires restarting the clojure
process. Normally, it does not require even that. In short, with
Tawny-OWL patterns become an integral part of ontology development,
rather than an external imposition.

In this chapter, we first describe how to use an pre-existing pattern
provided by Tawny-OWL, then how to modify this slightly for the
amino-acid ontology.  Finally, we show how to create a \textit{de
  novo} patternised section creating several hundred defined classes.

\section{Creating the Amino Acid Ontology}
\label{sec:creating-amino-acid}

First, we start with a namespace declaration. This is slightly different from
ones used before, as it also |require|s two new namespaces. |tawny.pattern|
provides pattern support and one key pattern which forms the core of the
amino-acid ontology; |clojure.string| provides string maninpulation
capabilities which we will use. We also define the new ontology.

\begin{tawny}
(ns take.wing.amino-acid
  (:import [org.semanticweb.owlapi.search EntitySearcher])
  (:require
    [clojure.string]
    [tawny.owl :refer :all]
    [tawny.pattern :as p]
    [tawny.reasoner :as r]
    [tawny.util :as u]))

(defontology aao
  :iri "http://www.purl.org/ontolink/aao")
\end{tawny}

First, to explain the domain. Proteins are polymers made up from amino-acid
monomers. They consist of a central carbon atom, attached to a carboxyl group
(the ``acid'' amino) and amine group (the ``amino'' group) a hydrogen and an R
group. The R group defines the different amino acids. The different R groups
have different phyiscal or chemical properties, such as their degree of
hydrophobicity. We call these different characteristics |RefiningFeatures|.

\begin{tawny}
(defclass AminoAcid)

(defclass RefiningFeature)
(defclass PhysicoChemicalProperty :super RefiningFeature)
\end{tawny}

There are a number of different ways of measuring hydrophobicity; in reality,
it is a continuous property rather than a discrete one, but these are hard to
model ontologically. One simple solution to this problem is the \emph{value
  partition} -- we just pick a set of discrete values into which we partition
the range. It is the same trick that is used to describe the colours of the
rainbow; we force a continuous range into seven colours. Hydrophobicity splits
into just two -- hydrophobicic and hydrophilic.

The full representation of this knowledge as a value parition is fairly
complex. First, we define a root class and an object property, with
an appropriate domain and range, and declared functional, as one object can be
hydrophilic or hydrophobicic but not both.

\begin{tawnyexample}
(defclass Hydrophobicity :super PhysicoChemicalProperty)

(defoproperty hasHydrophobicity :domain AminoAcid
  :range Hydrophobicity :characteristic :functional)
\end{tawnyexample}

Next we need to define the partition values. We make |Hydrophilic|
disjoint from |Hydrophobic|. We do not make |Hydrophobic| disjoint
from |Hydrophillic| because of Tawny-OWLs ``define before use''
semantics.

\begin{tawnyexample}
(defclass Hydrophobic :super Hydrophobicity)
(defclass Hydrophilic :super Hydrophobicity :disjoint Hydrophobic)
\end{tawnyexample}

Finally, we |refine| the first partition |Hydrophobic| to also be disjoint
with |Hydrophilic| and then add a covering axioms to |Hydrophobicity|.

\begin{tawnyexample}
(refine Hydrophobic
  :disjoint Hydrophilic)

(refine Hydrophobicity
  :equivalent (object-or Hydrophilic Hydrophobic))
\end{tawnyexample}

Of course, as we have already seen, the use of disjoints and covering
axioms is so common that Tawny-OWL provides specific support for
adding these, in a way which also allows us to avoid the necessity for
refining classes after creation. This produces a much neater
definition and is a simple example of the use of patterns.

\begin{tawnyexample}
(as-subclasses
    (defclass Hydrophobicity :super PhysicoChemicalProperty)
  :disjoint :cover
  (defclass Hydrophobic)
  (defclass Hydrophilic))

(defoproperty hasHydrophobicity :domain AminoAcid
   :range Hydrophobicity :characteristic :functional))
\end{tawnyexample}

This is, however, all still fairly long-winded and relatively easy to
get wrong.  Tawny-OWL, however, allows us to go further with the use
of the |defpartition| macro, which allows specification of all the
appropriate values at once. It will produce the same axioms as the
statements above.

\begin{tawny}
(p/defpartition Hydrophobicity
  [Hydrophobic Hydrophilic]
  :comment "The tendency to associate with water."
  :super PhysicoChemicalProperty
  :domain AminoAcid)
\end{tawny}

|defpartition| is a generic pattern and is not specific at all to the
amino-acid ontology. It will serve well, but for the amino-acid
ontology we need to define a series of further value partitions. They
all have the same super class and domain. It would be nice to create a
\emph{localised} pattern which hard-codes these values. As
|defpartition| is a macro this is slightly more complex than a normal
function, but not heavily so. This macro is unlikely to be of use in
another ontology because of these hard-coded values, but it is
valuable because it saves typing here and safe-guards us against
future changes. Being in the same environment, it is easy to do, so we
might as well!

\begin{tawny}
(defmacro defaapartition [& body]
  `(p/defpartition
     ~@body :super PhysicoChemicalProperty
     :domain AminoAcid))
\end{tawny}

The next value partition is as a result somewhat smaller, as it no
longer needs to describe the super class and domain. The size value
partition is self-explanatory enough; this could be described in
relation to a continuous physical measurement (such as size in
Daltons), but this is not necessary here.

\begin{tawny}
(defaapartition Size
  [Small Tiny Large]
  :comment "The physical size of the amino acid.")
\end{tawny}

Finally, we create three more value partitions describing |Charge|,
|SideChainStructure| and |Polarity|.

\begin{tawny}
(defaapartition Charge
  [Negative Neutral Positive]
  :comment "The charge of an amino acid.")

(defaapartition SideChainStructure
  [Aliphatic Aromatic]
  :comment "Does the side chain contain rings or not?")

(defaapartition Polarity
  [Polar NonPolar]
  :comment "The polarity across the amino acid.")
\end{tawny}

Next, we define a set of annotation properties. In the previous
Chapter~\ref{cha:pizza-ontology}, we made some use of a few annotation
properties: the label and the comment. But, in OWL, annotation
properties are generic. It is possible to define new annotation
properties. This is useful here because amino-acids have a long name,
such as |Alanine|, and two shorter names -- a three letter abbreviation such
as |Ala| and finally one letter abbreviation which is shorter, but harder to
remember, in this case |A|. These abbreviations are standardized and
widely used, so worth describing here.

\begin{tawny}
;; annotation properties
(defaproperty hasLongName)
(defaproperty hasShortName)
(defaproperty hasSingleLetterName)
\end{tawny}

Now, we move onto the heart of this amino-acid ontology which is the function
which defines a single amino-acid. This is a fairly large definition, but it
is fairly repetitive in itself. First we start with the function definition,
combined with a few small pre-conditions; these are probably unnecessary in
this case, for reasons we will see soon.

\begin{tawny}
(defn amino-acid
  "Define a new amino acid. Names is a vector with the long, three letter and
  single amino acid version. Properties are the five value partitions for each
  aa, as a list."
  [names properties]
  {:pre [(= 3 (count names))
         (= 5 (count properties))]}
\end{tawny}

The main part of the amino acid pattern is defined in the next section. The
pattern is not that complex -- we simply give an amino-acid five properties
and three names. However, to achieve this, we use a new feature of
Tawny-OWL: the |gem| and the |facet|.

The |defpartition| macro that we introduced earlier \todo{Aslo does facets}

We introduce here a new pattern function called |gem|.

 This is done inside a |let| block because we want to capture the return
value. This is not strictly necessary as the return value is used only once,
but in this case, I think, it increases readability.

\begin{tawny}
  (let [aa (p/gem (first names)
                :super AminoAcid
                ;; we have don't test the values are correct here
                ;; because the code layout should make the order obvious
                ;; and the range constraints should protect us during
                ;; reasoning.
                :facet properties
                :label (first names)
                :annotation
                (annotation hasLongName (nth names 0))
                (annotation hasShortName (nth names 1))
                (annotation hasSingleLetterName (nth names 2)))]
\end{tawny}

The last part is not part of the pattern itself. Rather it adds support for
\emph{interning}; this is the process by which OWL objects are bound to
Clojure symbols. The practical upshot of this is that we (or anyone importing
the amino acid ontology) will be able to refer to amino acids using names like
|Alanine| rather than being required to use strings inside quotes ---
|"Alanine"|. This adds (considerable) complexity to the Tawny-OWL definition
of the amino-acid ontology, but is probably worth it for ease of downstream
use.

To achieve this, we need to return instances of the |tawny.pattern.Named|
class, combined with the strings we use to refer to them. In this case, a
single amino-acid class gets three names -- this is rather unusual but makes
sense here.

\begin{tawny}
    ;; and return types for intern
    (map p/->Named
         names
         (repeat aa))))
\end{tawny}

We could stop here in terms of generating our ontology. However, here we take
two more steps, one mostly to make the input more consistent, so that we would
see errors easily, and one to make the amino-acid ontology more usuable within
the Tawny-OWL environment.

Firstly, we define a function which takes a number of different amino-acid
definitions and runs the amino-acid function over them. It then flattens the
list of lists that is returned.

\begin{tawny}
(defn amino-acids
  [& definitions]
  (apply
   concat
   (map
    (fn [[names props]] (amino-acid names props))
    (partition 2 definitions))))
\end{tawny}

Finally, we define a macro. This does two things for us. Firstly it
provides the convienience of using ``bear'' words: so |Alanine|
instead of |"Alanine"| within the macro itself. A small convienience
for a single amino-acid, but a bigger one for all twenty. There are a
variety of ways of achieving this -- we could use the
|tawny.util/quote-tree| macro to covert all the symbols to
strings. However, here, we take the slightly more complex route and
just turn the first part of the definition into strings. The rest can
remain symbols as they are pre-defined.  And, secondly, we
\emph{intern} the |Named| values turned from the |amino-acid|
function; that is we create a new variable, identified by relevant
symbol, with a value which is an OWL entity. The practical upshort of
this is that later, we can refer to |Alanine| (or |Ala| or |A|)
rather than having to use quotes. In terms of the amino-acid ontology
itself, this is unnecessary, but it is useful for another ontology
importing the amino-acid ontology, so it is worth doing here. In
addition and probably more importantly than the convienience, this
also provides a degree of safety: attempts, for instance, to refer to
an amino-acid |B| will fail with an error as this amino-acid does not
exist.

\begin{tawny}
(defmacro defaminoacids
  [& definitions]
  (let [definitions
        (interleave
         (map
          #(mapv name %)
          (take-nth 2 definitions))
         (take-nth 2 (rest definitions)))]
    `(p/intern-owl-entities
      (amino-acids ~@definitions))))
\end{tawny}

Then, we define all the amino-acid. These have been laid out in
alphabetical order, and the properties arranged in a table which means that we
can visually check that everything is correct and nothing is missing.

\begin{tawny}
(defaminoacids
  [Alanine       Ala A] [Neutral  Hydrophobic NonPolar Aliphatic Tiny]
  [Arginine      Arg R] [Positive Hydrophilic Polar    Aliphatic Large]
  [Asparagine    Asn N] [Neutral  Hydrophilic Polar    Aliphatic Small]
  [Aspartate     Asp D] [Negative Hydrophilic Polar    Aliphatic Small]
  [Cysteine      Cys C] [Neutral  Hydrophobic Polar    Aliphatic Small]
  [Glutamate     Glu E] [Negative Hydrophilic Polar    Aliphatic Small]
  [Glutamine     Gln Q] [Neutral  Hydrophilic Polar    Aliphatic Large]
  [Glycine       Gly G] [Neutral  Hydrophobic NonPolar Aliphatic Tiny]
  [Histidine     His H] [Positive Hydrophilic Polar    Aromatic  Large]
  [Isoleucine    Ile I] [Neutral  Hydrophobic NonPolar Aliphatic Large]
  [Leucine       Leu L] [Neutral  Hydrophobic NonPolar Aliphatic Large]
  [Lysine        Lys K] [Positive Hydrophilic Polar    Aliphatic Large]
  [Methionine    Met M] [Neutral  Hydrophobic NonPolar Aliphatic Large]
  [Phenylalanine Phe F] [Neutral  Hydrophobic NonPolar Aromatic  Large]
  [Proline       Pro P] [Neutral  Hydrophobic NonPolar Aliphatic Small]
  [Serine        Ser S] [Neutral  Hydrophilic Polar    Aliphatic Tiny]
  [Threonine     Thr T] [Neutral  Hydrophilic Polar    Aliphatic Tiny]
  [Tryptophan    Trp W] [Neutral  Hydrophobic NonPolar Aromatic  Large]
  [Tyrosine      Try Y] [Neutral  Hydrophobic Polar    Aromatic  Large]
  [Valine        Val V] [Neutral  Hydrophobic NonPolar Aliphatic Small]
  )
\end{tawny}

Finally, we clean up by ensuring that all aminoc-acids are disjoint from each
other. We could do this earlier in the |amino-acids| function, but as this
function only needs to be run once, it makes little difference.

\begin{tawny}
(apply as-disjoint (subclasses AminoAcid))
\end{tawny}

\section{Defining the Amino Acids}
\label{sec:defining-amino-acids}

We saw earlier, while consider pizza, that it is possible to create
defined classes (see Section~\ref{defined}). For the amino-acid
ontology, this is also tremendously useful because we can effectively
use this to query it. Consider, for example, this definition of
|LargeAminoAcid|.

\begin{tawnyexample}
(defclass LargeAminoAcid
  :equivalent (owl-some hasSize Large))
\end{tawnyexample}

This is fine, of course, but is also very slow, as there are a lot of
potential classes that we could create. As well as one for each of the twelve
values in our five value partitions, we also need all of the permutations of
these, which makes quite a few classes.

Of course, being fully programmatic, calculating permutations in Tawny-OWL is
a simple enough task; so, why not build all of these defined classes
programmatically?

This is reasonably straight-forward; first, we need a definition for a
defined class; this will take a list of partition values. The pattern
simply involves making existential (|owl-some|) restrictions to all of
the partition values using the appropriate object property; we can
achieve this using the |facet| function that we saw earlier. We form
the name of the class from the names of the partition values.

\begin{tawny}
(defn amino-acid-def [partition-values]
  (let [name
        (str
         (clojure.string/join
          (map
           #(.getFragment
             (.getIRI %))
           partition-values))
         "AminoAcid")
        exist (p/facet partition-values)]
\end{tawny}

Then finally we create the class and package it with its name. As with our
previous amino-acid definition, this function has a return value which would
allow it to be used to intern the classes created, although we do not actually
use that facility here.

\begin{tawny}
    (p/->Named
     name
     (owl-class
      name
      :label name
      :equivalent
      (owl-and AminoAcid exist)))))
\end{tawny}

Calculating a cartesian product is relatively easy in Clojure using the
swiss-army knife |for| list comprehension.

\begin{tawny}
(defn cart [colls]
  (if (empty? colls)
    '(())
    (for [x (first colls)
          more (cart (rest colls))]
      (cons x more))))
\end{tawny}

We combine all of these together to create all of the defined classes.

\begin{tawny}
;; build the classes
(doall
 (map
  amino-acid-def
  ;; kill the empty list
  (rest
   (map
    #(filter identity %)
    ;; combination of all of them
    (cart
     ;; list of values for each partitions plus nil
     (map
      #(cons nil (seq (direct-subclasses %)))
      ;; all our partitions
      (seq (direct-subclasses PhysicoChemicalProperty))))))))
\end{tawny}

Finally, we check to see whether everything has worked. For this, we will need
to use a reasoner, so first we choose a reasoner and check the consistency of
our ontology.

\begin{tawny}
(r/reasoner-factory :hermit)
(r/consistent?)
\end{tawny}

We can also investigate the classes that we have created. None of the created classes
should have any asserted subclasses, which we can check, by looking at one.

\begin{tawny}
(subclasses
 (owl-class "SmallAminoAcid"))
\end{tawny}

However, we see a totally different picture with the reasoner. We can first
check for inferred subclasses.

\begin{tawny}
(r/isubclasses
 (owl-class "SmallAminoAcid"))
\end{tawny}

We might have expected to just see a few as there are only 20 amino-acids, but
actually, there are 113 of them. The reason for this is that the reasoner
determines the subclass relationships between the defined classes as well as
with the named amino-acids: so, for instance, an |HydrophobicSmallAminoAcid|
is necessarily also a |SmallAminoAcid| so appears as a subclass. This
demonstrates the power of using a computational reasoner; while the
conclusions that it reaches are not, in this case, difficult to calculate by
hand, with so many classes they would be laborious.

Unfortunately, in this case, they also hide the answer that we are really
interested in. In a less programmatic tool, we would be stuck, but this is not
a problem in Tawny-OWL; we just filter the defined classes from the result as
follows.

\begin{tawny}
(filter
 #(not (EntitySearcher/isDefined % aao))
 (r/isubclasses
  (owl-class "SmallAminoAcid")))
;; => (#[Class 0x6f148b89 "Valine"@en] #[Class 0x4c949f3e "Proline"@en]
;;     #[Class 0x57f95b76 "Glutamate"@en] #[Class 0x784dcd0 "Asparagine"@en]
;;     #[Class 0x4c07576a "Aspartate"@en] #[Class 0x24024cbb "Cysteine"@en])
\end{tawny}

And the end result? There are six small amino-acids!

With Tawny-OWL it is straight-forward to implement new patterns building a
very large number of classes at once; the amino acid ontology is a nice
example of this. At the current time, we do not really know how common the
requirement is for this sort of ontology; most ontologies in existance are not
heavily patternized. But, then, perhaps this is part because the tools for
generating patterns were not integrated into our ontology development process;
patternized ontologies are not common because they are just too painful to
produce.

Even aside from heavily patternized ontologies, this chapter also shows that
Tawny-OWL can be easily extended even within the scope of a single ontology.
The |defaapartition| macro is only useful here. But, it is easy to write,
reduces duplication and increases consistency of the end ontology. Most
ontologies have this form of repetition. With Tawny-OWL, managing this
repetition becomes the task of the computer and not the task of the human,
which is as it should be.

\section{Recap}
\label{sec:recap-2}

In this chapter, we have described:

\begin{itemize}
\item The |tawny.pattern| namespace.
\item The Value-Partition design pattern
\item Gems and Facets
\item A macro expanding the value-partition.
\item An amino-acid function
\item Intern with |intern-owl-entities|
\item A highly patternized part of the ontology.
\end{itemize}
