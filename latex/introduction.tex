\chapter{Introduction}
\label{cha:introduction}

This book introduces ontology building using the OWL2 ontology language, and
the Tawny-OWL library. Ontologies are a method for representing knowledge,
generally, but not necessarily, about the world around us. It is then possible
to check that the representation is consistent, as well as drawing conclusions
about new knowledge. They are generally used in complex, knowledge-rich areas
of knowledge, including biomedicine.

Many ontology development tools provide a Graphical User Interface, through
which the ontology developer adds the various entities involved in building an
ontology. However, many ontologies contain large and repetitive sections; for
these, ontology development teams often fall back to generating parts of their
ontology programmatically. Tawny-OWL takes a different approach where ontology
development in a domain-specific language (DSL) embedded in a full programming
language. For structurally simple parts of an ontology, the various components
of an ontology can be specified using the default convienient and simple
Tawny-OWL syntax; for structurally complex parts, new syntax and new patterns
can be built, extending the environment as a core part of ontology
development.

This form of programmatic ontology development is still young. At the moment,
we have used it to produce large ontologies that would have been difficult
using any other technique. However, we also hope that we can also support
easier integration of knowledge-rich structures into applications, so that
ontological data structures can be come a standard part of the programmers
toolkit.


\section{Status}
\label{sec:status}

This manual describes Tawny-OWL 2.0. While more Chapters are planned,
it now describes all the key features of this Tawny-OWL environment.

It is hosted at \href{https://phillord.github.io/take-wing/}.



\section{What is an Ontology}
\label{sec:what-an-ontology}

Ontologies are about definitions. It is, perhaps, unsurprising therefore
that amount ontologists there are quite a few debates about what exactly
an ontology is and is not; it is not our intention here to either cover
these arguments, nor to give a comprehensive overview of all the uses of
the word.

What is generally agreed is that ontologies describe a set of entities,
in terms of the relationships between these entities, using any of a
number of different relationships. So, for example, we can describe
entities in terms of their class relationships -- what is true of a
superclass is also true of all subclasses. Or we can describe the
\emph{partonomic} relationships: the finger is part of the hand, which is
part of the foot.

An ontology is also very similar to a taxonomy; however, ontologies
place much greater emphasis on their computational properties. This
makes ontologies much more suitable for driving applications and code,
although this often comes at the cost of human understandability of the
ontology. In this document, all the ontologies we talk about are
represented using specific language, called OWL (the Ontology Web
Language). This has very well-defined computational properties, and
through the document we will explore the implications of these
properties.

We also use the term "ontology" to mean a specific object that you can
manipulate in Tawny-OWL, which is a slightly more constrained use. This is
quite common in many books on programming: we hope that the context should be
clear.

\section{Who this book is for}
\label{sec:who-this-book}

We have two primary audiences for this book. The first is for the ontologist
who is interested in Tawny-OWL as the hub of a new environment for ontology
development. The second is for the programmer who is interested in using the
rich computational representation of a domain that ontologies provide. There
is a risk to having two audiences: that we satisfy neither. To avoid this, the
book is built from a series of chapters, each of which covers a discrete
topic, either more programmatic or more ontological. It should be possible to
read the chapters independently of each other.

This book does not, however, stand alone. While we try to introduce the
background material, we do not intend that, for example, this book will serve
as an introduction to programming either in general, or specifically in
\index{Clojure}Clojure. There are many good resources available for this. With
ontologies, we give more of a background introduction, but again, we assume
that you will be willing to read other material to clarify ontology
development. Our hope is that we introduce the material well enough that you
feel it is worth the time to investigate other resources, and we include
pointers where it seems valuable.




