\chapter{An Introduction to OWL}

In Section\ref{sec:what-an-ontology}, we briefly touched on the issue
of what is an ontology and noted that it's not easy. In this section,
we will take a more pragmatic view point and describe OWL and its
notion of an ontology.

OWL2 is the second version of the Ontology Web Language; it is a W3C
recommendation\footnote{W3C is the body that would define standards
  for the Web, if it makes standards. Except that it does not; it just
  releases recommendations.}. As you might expect, this means that it
embeds well with other W3C standards -- it can be serialized as XML or
RDF, and it makes quite intensive use of URLs.

 It also builds on many years of Computing research; underneath each of
the statements that we can make in OWL is a mapping to a piece of
formal maths which gives a tightly defined \emph{semantics}. We will
only touch of this semantics lightly in this document\footnote{Mostly
  because if I touched on it more heavily, I'd probably get it wrong};
the key point is that this specification makes it possible to build
software around OWL and have it come to a clearly defined conclusions.

Using the statements in OWL we can build models of the world. That is
we can describe the real world around us using statements in OWL; as a
result, we can use these underlying semantics of OWL to draw
conclusions about these models. If we do this right, these conclusions
should also be true of the real world as well.

There are a number of different ways that we could build models, but
OWL does this with three entities: individuals, classes and
properties. In addition, to enable OWL ontologies to describe the real
world, it also has two further entities: identifiers and annotations.


\section{Individuals}
\label{sec:individuals}

At heart, OWL ontologies describe a set of individuals. In the real
world, these would be the things that we want to describe. Looking
around me now, I can see a large number of these things: a computer
screen (obviously); a keyboard; assorted other pieces of computing
detritus; a guitar; a door; and, finally, somewhat incongruously, a
toilet seat. Individuals in OWL can also describe more abstract things
such as the image on my screen, the process of me typing and so forth.

Sometimes, individuals are also called instances; we do not use this
term here, because it causes confusion with people who come
Object-Oriented programming background where it has a related but
subtly different meaning\footnote{Like ``object'' which also has an
  ontological meaning.}.

In OWL ontologies, individuals also have a name or an
identifier\footnote{Although, some individuals are anonymous. We will
  discuss more on the form of identifiers later.}. Actually, they can
any number of names and, perhaps, unintuitively, OWL will not assume
that they have an unique name; so, unless you tell it explicitly, OWL
will not know whether two different identifiers describe two different
individuals with one name each, or one individual with two names.

\section{Properties}
\label{sec:properties-1}

Individuals can have relations between them. In OWL, these are called
properties\footnote{Roughly equivalent to properties or attributes in
  OO terminology}. So, the |I| and |typing| on my |Keyboard|. In OWL
properties are \emph{binary} -- that is they only describe a
relationship between two individuals\footnote{This is less restrictive
  than it sounds.}. Properties in OWL have a number of
characteristics, which we will describe later.

It is also possible to use properties to describe a relationship
between an individual and something \emph{concrete} -- such as a
numeric value or a string.

\section{Classes}
\label{sec:classes}

Classes in OWL are \emph{sets} on individuals. All the individuals in
a class will share some of the same characteristics. Classes have
relationships between themselves which turn them into a hierarchy. So,
both my |Trackball|, |Keyboard| and |Monitor| are subclasses of
|Peripherals|. In OWL, the meaning of the subclass relationship is
quite specific -- if |A| is a subclass of |B|, then all individuals of
class |A| are also individuals of class |B|.

For people coming from an programming background, this looks very like
object-orientation (OO) and its notion of instances, classes and
subclasses. But there is subtle, but important difference. In OO,
instances are explicitly stated to be part of a class, and inherit
properties from this class. In OWL, it is the other way around:
individuals have properties, and then properties that they have define
the classes that they are in. We can see this in
Section~\ref{sec:taster}, where we can \emph{infer} that |TakeWing| is
an |OntologyBook|.

\section{Identifiers}

To make all of the logical entities in OWL useful, we need
\emph{identifiers} which allows us to refer to them. Again, most
programming languages have this sort of capability: variable names,
class names and so forth. OWL is rather different here and shows its
web heritage; it uses IRIs for identifiers\footnote{IRIs are not the
  same thing as URIs, which are not the same thing as URLs. But the
  differences between them are relative unimportant here.}.

Identifiers in OWL, therefore, are effectively universal; a class in
one ontology can unambiguously refer to a class in another. More over,
it can use and share identifiers described and defined in all the
other web technologies.

Tawny-OWL maps these identifiers on to its own which inherits from its
base language of Clojure; this largely stems from the requirements for
identifiers which are easy to type and use.


\section{Annotations}

Those who are interested in the underlying semantics of OWL often
describe annotations are \emph{extra-logically}. This rather
downgrades their importance; it is annotations that allow the
underlying logic to relate to the real world around. The underlying
logic of OWL may provides predictable behaviour, but is the
annotations which provide all the utility of an OWL ontology, by
relating to the real world and to the user.

OWL allows annotations on pretty much anything. Classes, individuals
and properties can all have annotations; the axioms that assert these
entities can have annotations; annotations can have annotations; it is
even possible to use annotations to provide descriptions of why
annotations have annotations. It is entirely possible that the
designers of OWL got a little carried away with annotations, Tawny-OWL
supports the many different forms of annotation anyway.
