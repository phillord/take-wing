\chapter{Getting Started}
\label{cha:getting-started}

\begin{tawnyhidden}
(ns take.wing.getting-started)
\end{tawnyhidden}

In this section, we will build the most ontology and start to show the
basic capabilities of Tawny-OWL.

As described in \label{/the/environment-the-environment}, Tawny-OWL can be
used with several different toolchains. In this section, we will run
through the building a very simple ontology. There is an section describing
how to achieve each of these steps with specific tool chains.

\subsection{Installing Leiningen}
\label{sec:installing-leiningen}

To build an ontology, we need a build tool, for which we will use
\url{https://leiningen.org}{leiningen}. This is a command line
application and is simple to install following the instructions on
their website.

Installing leiningen is the only manual step involved. It is leiningen
that is responsible for everything else; it downloads Tawny-OWL and
all of its dependencies for you.

\subsection{Creating a New Project}
\label{sec:creating-new-project}

Now, we will create a new project. Tawny-OWL makes this easy with a
pre-defined template.

\begin{verbatim}
lein new ontology helloworld
\end{verbatim}

This will create a new directory called |helloworld|. If we change
into this directory, we find that this has created a number of
directories and files.

Before we look in more detail at these files, let start by generating
an ontology file. Simply type:

\begin{verbatim}
lein run
\end{verbatim}

You should see that a new file has been created called
\verb|helloworld.omn| which contains a very simple ontology with a
single class called |HelloWorld|.

\subsection{Editing Our Ontology}
\label{sec:editing-our-ontology}

Tawny-OWL provides a fully programmatic development environment for
ontologies; as such, it is possible to change or update an ontology
with an editor or any IDE. In this section, we will use a simple,
web-based editor that integrates tightly with leiningen.

To use this try:

\begin{verbatim}
lein with-profile light nightlight
\end{verbatim}

This should return something like:

\begin{verbatim}
Started Nightlight on http://localhost:4000
\end{verbatim}

Open this address in a web-browser and you should now be able to see
the editor. This in turn will enable you to look at the Tawny-OWL files.

First, we consider the file |helloworld.clj|; this looks like so:

\lstinputlisting[style=tawnystyle]{../new-project/helloworld/src/helloworld/helloworld.clj}

Breaking this down. We first start with by introducing the namespace
and |use|ing Tawny-OWL. These identical statements appear at the
beginning of every Tawny-OWL file: the namespace introduced must match
the file name.

Next, we create a new ontology called |helloworld|, with a single
class also called (somewhat repetitively), |HelloWorld|. Tawny-OWL is
case-sensitive, so these two things are independent from each other.

The second file, |core.clj| is more programmatic in nature. It
|require|s |helloworld|, and then defines a function called |-main|
which saves the ontology.

\lstinputlisting[style=tawnystyle]{../new-project/helloworld/src/helloworld/core.clj}

The practical upshot of this all taken together is that typing

\begin{verbatim}
lein run
\end{verbatim}

at the command line will result in a new file (called
|helloworld.omn|) with an ontology in OWL Manchester Notation (OMN).

\subsection{Summary}
\label{sec:summary}

In this section, we have outlined the basic tasks that are needed to
build ontologies with Tawny-OWL: creating a project, creating an
ontology, creating some entities. We have also started to show how to
use and query over them. In the next section, we will build this
ontology in full, using it to demonstrate many parts of Tawny-OWL and
OWL ontologies in general.

