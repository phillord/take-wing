\chapter{Getting Started}

\section{Environment}
\label{sec-3}

Our hope is that for structurally simple ontologies, Tawny-OWL should be
usable by non-programmers, with a simple and straight-forward syntax.
One area where this hope is currently not fulfilled is right at the
start -- getting a working environment is not as simple as starting an
application such as Protege and programming. In this section, I
introduce the core technology and the basic environment that is needed
to make effective use of Tawny-OWL.


\subsection{The OWL API}
\label{sec-3-1}

Tawny-OWL is built using the \url{http://owlapi.sourceforge.net/[OWL} API].
This library is a comprehensive tool for generating, transforming and
using OWL Ontologies. It is widely used, and is the basis for the
Protege 4 editor. Being based on this library, Tawny-OWL is reliable and
standard-compliant (or at least as reliable and standard-compliant as
Protege!). It is also easy to integrate directly with other tools
written using the OWL API.

\subsection{Clojure}
\label{sec-3-2}

Tawny-OWL is a programmatic library build on top of the Clojure
language. Tawny-OWL takes many things from Clojure. These include:

\begin{itemize}
\item the basic syntax with parentheses and with \texttt{:keywords}
\item the ability to effectively add new syntax
\item the ability to extend Tawny-OWL with patterns
\item integration with other data sources
\item the test environment
\item the build, dependency and deployment tools
\end{itemize}

In addition, most of the tools and environment that Tawny-OWL use to
enable development were built for Clojure and are used directly with
little or no additions. These include:

\begin{itemize}
\item IDEs or editors used for writing Clojure
\item the leiningen build tool
\end{itemize}

Tawny-OWL inherits a line-orientated syntax which means that it works
well with tools written for any programming language; most notable
amoung these are version control systems which enable highly
collaborative working on ontologies.

Clojure is treated as a programmatic library -- the user never starts or
runs Clojure, and there is no \texttt{clojure} command. Rather confusingly,
this role is fulilled by Leiningen, which is the next item on the list.

\subsection{Leiningen}
\label{sec-3-3}

\url{http://www.leiningen.org[Leiningen}] is a tool for working with Clojure
projects. Given a directory structure, and some source code leiningen
will perform many project tasks including checking, testing, releasing
and deploying the project. In addition to these, it has two critical
functions that every Tawny-OWL project will use: first, it manages
dependencies, which means it will download both Tawny-OWL and Clojure;
second, it starts a REPL which is the principle means by which the user
will directly or indirectly interact with Tawny-OWL.

\subsection{REPL}
\label{sec-3-4}

Clojure provides a REPL -- Read-Eval-Print-Loop. This is the same things
as a shell, or command line. For instance, we can the following into a
Clojure REPL, and it will print the return value, or 2 in this case.


\begin{tawny}
;; returns 2 
(+ 1 1)
\end{tawny}

The most usual way to start a REPL is to use leiningen, which then sets
up the appropriate libraries for the local project. For example,
\texttt{lein repl} in the source code for this document, loads a REPL with
Tawny-OWL pre-loaded.

In practice, most people use the REPL indirectly through their IDE.

\subsection{IDE or Editor}
\label{sec-3-5}

Clojure is supported by a wide variety of editors, which in turn means
that they can be used for Tawny-OWL. The choice of an editor is a very
personal one (I use Emacs), but in practice any good editor will work.

The editor has two main roles. Firstly, as the name suggests it provides
a rich environment for writing Tawny-OWL commands. Secondly, the IDE
will start and interact with a REPL for you. This allows you to add or
remove new classes and other entities to an ontology interactively.
Tawny-OWL has been designed to take advantage of an IDE environment; in
most cases, for example, auto-completion will happen for you.

\subsection{Further Information}
\label{sec-3-6}

There are many sources of further information about Clojure which will
be listed here.



\section{Getting Started}
\label{sec-4}

In this section, we will build the most ontology and start to show the
basic capabilities of Tawny-OWL.

As described in \label{/the/environment-the-environment}, Tawny-OWL can be
used with several different toolchains. In this section, we will run
through the building a very simple ontology. There is an <> describing
how to achieve each of these steps with specific tool chains.

\subsection{Getting a Project}
\label{sec-4-1}

For this book, we will use a pre-rolled project -- in fact the one used
to create this book. You can access the project data from
\url{https://github.com/phillord/take-wing[github}], either using \texttt{git} or
through the download option. If you wish to know how to build a project
yourself, please read <>.

A leiningen project is, essentially, a directory structure with a
project file. The \texttt{project.clj} file for this book looks like this:

\begin{clojure}
(defproject take-wing "0.1.0-SNAPSHOT"
  :dependencies [[uk.org.russet/tawny-owl "1.1.1-SNAPSHOT"]])
\end{clojure}

This includes three critical pieces of information. Firstly \texttt{take-wing}
which is the name of the project. Secondly, immediately after this is a
version number such as \texttt{0.1.0}. Finally, we have a \texttt{:dependencies} which
includes only a single dependency to \texttt{tawny-owl} itself.


\subsection{Starting a new ontology}
\label{sec-4-2}

It is possible to build an ontology in Tawny-OWL using almost no
functions from Clojure, the language on which it is built; the only
necessary exception to this is the \emph{namespace declaration}.

Like most programming languages, Clojure has a namespacing mechanism.
These are declared at the start of each file, and the namespace relates
to the file name and location. Finally, the namespace form is also used
to import functions from other namespaces. Here, we define a namespace
called \texttt{take.wing.getting-started} which would be defined in a file
\texttt{take/wing/getting\_started.clj}. Secondly, we import \texttt{tawny.owl}
namespace which contains the core functions of Tawny-OWL.

\begin{tawny}
(ns take.wing.getting-started (:use [tawny.owl]))
\end{tawny}

Tawny-OWL also uses the namespace mechanism to define the scope of an
ontology. In general, an ontology is defined within a single namespace,
and each namespace defines a single ontology. A new ontology is declared
with the \texttt{defontology} form. This also introduces a new symbol, \texttt{pizza},
which can be used latter to refer.

\begin{tawny}
(defontology pizza)
\end{tawny}


\subsection{Connecting to a repl}
\label{sec-4-3}

Currently, the source for the ontology has been created, but this is not
"live" -- for this, we must start a Clojure process and connect to it
via a REPL and then evaluate the file. With the current contents, the
REPL should show something like the following which is the result of
evaluating the last form, and shows that we have defined a new symbol
footnote:[In Clojure, it is actually a var that has been created]

\begin{verbatim}
=> #'take.wing.getting-started/pizza
\end{verbatim}

The symbol \texttt{pizza} now refers to an object live in the system. If
evalulate \texttt{pizza}, a hopefully informative string message will be
printed.


\begin{verbatim}
take.wing.getting-started> pizza
#<OWLOntologyImpl Ontology(OntologyID(OntologyIRI(<#pizza>))) [Axioms: 0 Logical Axioms: 0]>
\end{verbatim}


\subsection{Creating some entities}
\label{sec-4-4}

Now we create some entities for our ontology, in this case two classes
called \texttt{Pizza} and \texttt{MargheritaPizza}, and state that \texttt{MargheritaPizza}
is a subclass of \texttt{Pizza}. This forms implicitly place these two terms


\begin{tawny}
(defclass Pizza)
(defclass MargheritaPizza :super Pizza)
\end{tawny}

Finally, we are in a position to make a useful query against this which
we can do using the \texttt{subclasses} function.

\begin{tawny}
(subclasses Pizza)
\end{tawny}

In a REPL session, this returns a set with one element ---
\texttt{MargheritaPizza}. If we evaluate \texttt{pizza} (the ontology) again, we also
see that the ontology now has a number of axioms.


\begin{verbatim}
(subclasses Pizza)
#{#<OWLClassImpl <#pizza#MargheritaPizza>>}
take.wing.getting-started> pizza
=> #<OWLOntologyImpl Ontology(OntologyID(OntologyIRI(<#pizza>))) [Axioms: 5 Logical Axioms: 1]>
\end{verbatim}


\subsection{Summary}
\label{sec-4-5}

In this section, we have outlined the basic tasks that are needed to
build ontologies with Tawny-OWL: creating a project, creating an
ontology, creating some entities. We have also started to show how to
use and query over them. In the next section, we will build this
ontology in full, using it to demonstrate many parts of Tawny-OWL and
OWL ontologies in general.

